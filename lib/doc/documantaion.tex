\documentclass{article}
% This is czech documantion of my project "JavaMath". 
% Here I'll explain everything that made this projet possible, every site that I used to gather infomation etc.
% Packages
\usepackage{listings}
\usepackage{color}
\usepackage{mathtools}
% \usepackage{hyperref}
% Colors
\definecolor{dkgreen}{rgb}{0,0.6,0}
\definecolor{gray}{rgb}{0.5,0.5,0.5}
\definecolor{mauve}{rgb}{0.58,0,0.82}
% Title page
\title{\textbf{JavaMath documentace}}
\author{Petr Kučera}
\date{\today}
% Pagenumbering
\pagenumbering{gobble}
% Sets listing for java code
\lstset{frame=tb,
  language=Java,
  aboveskip=3mm,
  belowskip=3mm,
  showstringspaces=false,
  columns=flexible,
  basicstyle={\small\ttfamily},
  numbers=none,
  numberstyle=\tiny\color{gray},
  keywordstyle=\color{blue},
  commentstyle=\color{dkgreen},
  stringstyle=\color{mauve},
  breaklines=true,
  breakatwhitespace=true,
  tabsize=3
}
% Main document
\begin{document}
    % \maketitle
    % \tableofcontents
    %% BigDecimalMath section
    \setcounter{section}{1}
    \section{BigDecimalMath} 
    V tomto souboru řeším matematické funkce, které nejsou normálně dostupné pro
    java modul \textit{BigDecimal}. Řešení těchto funkcí je vybráno tak, aby bylo co
    nejvíce nejpřesnější a jednoduše na provedení, ale zároveň aby čas vypočtení byl co
    nejmneší.

    %%% Subsection of constants and numbers
    \subsection{Konstanty a čísla}
    Součástí \textit{BigDecimalMath} jsou i konstanty a jiná užitečná čísla, pro rychlejší
    a přehlednější programování. 
    \textit{BigDecimal} modul pro Javu již nějaké takovéto čísla obsahuje, 
    například \textit{BigDecimal.ZERO} a \textit{BigDecimal.ONE}. Ovšem jiné variace, například
    pro číslo dva si musíme definovat vlastnoručně, proto jsem někerá tato čísla definoval pro
    jednoduší použití v \textit{BigDecimalMath}. 
    %%%% Code preview
    \begin{lstlisting}
        import java.math.BigDecimal;
        import java.math.RoundingMode;
        // Minus jedna
        public static final BigDecimal MINUSONE = new BigDecimal(-1);
        // Dva
        public static final BigDecimal TWO = new BigDecimal(2);
        // Pi hodnota - napsana na 1000 desetinnych mist
        public static final BigDecimal PI = new BigDecimal(
            "3.141592653589793238462643383279502884197169399..."
        );
        // E hodnota - napsana na 1000 desetinnych mist
        public static final BigDecimal E = new BigDecimal(
            "2.718281828459045235360287471352662497757247093..."
        );
        // 2*Pi
        pubic static final BigDecmal TWOPI = PI.multiply(TWO);
        // Pi/2
        public static final BigDecimal HALFPI = PI.divide(TWO, 1000, RoundingMode.HALF_UP);
        // Minus Pi/2
        public static final BigDecimal MINHALFPI = MINUSONE.multiply(HALFPI);
    \end{lstlisting}
    %%% Trigonometric subsection
    \subsection{Trigonometrické a Hyperbolické funkce}
    Původní použití pro tento soubor bylo použití při implementaci complexních čísel a 
    jejich funkcí do Java modulu \textit{BigDecimal}. Použití modulu \textit{Math} nebylo
    na místě, kvůli jeho omezení na \textit{double} hodnoty. Z tohoto důvodu jsem musel napsat
    nové algoritmy na výpočty trigonometrických funkcí, které jsem potřeboval použít. \newline
    Tento script neobsahuje všechny trigonometrické či hyperbolické funkce, jen ty, které jsem potřeboval.
    %%%% Sine approximation
    \subsubsection{Sinus}
    Pro aproximaci sinu z čísla jsem použil taylorovu sérii, která vypadá takto:
    \begin{center}
        $sin(x) = \displaystyle\sum_{n=0}^{\infty} \frac{(-1)^n}{(2x+1)!}x^{2x+1}$.
    \end{center}
    Výsledný algoritmus vypadá tedy takto:
    \sloppy
    \begin{lstlisting}
        public static BigDecimal sin(BigDecimal x) {
            BigDecimal ans = new BigDecimal(0);
            // Suma
            for (BigDecimal n = BigDecimal.ZERO; n.compareTo(new BigDecimal(50)) <= 0; n = n.add(BigDecimal.ONE)) {
                // Citatel
                BigDecimal numerator = MINUSONE.pow(n.intValue());
                // Jmenovatel
                BigDecimal denominator = factorial((TWO.multiply(n)).add(BigDecimal.ONE));
                // Vzorec uvnitr sumy
                ans = ans.add((numerator.divide(denominator, 1000, 
                    RoundingMode.HALF_UP)).multiply(
                    x.pow(((TWO.multiply(n)).add(
                    BigDecimal.ONE)).intValue())));
            }
            // Vraceni vysledku
            return ans.setScale(50, RoundingMode.HALF_UP);
        }
    \end{lstlisting}
    Tento algoritmus je nastaven na přesnost okolo $1\times{10^{-50}}$ s časem vypočítání okolo 10ms. Pro
    větší přesnost zvyšte \textit{n.compareTo(new BigDecimal(50))} hodnotu na vyšší.
    \subsubsection{Cosinus}
    Pro aproximaci cosinu je opět použita taylorova série, která vypadá takto:
    \begin{center}
        $cos x = \displaystyle\sum_{n=0}^{\infty} \frac{(-1)^n}{(2n)!}x^{2x}$
    \end{center}
    Výsledný algoritmu vypadá tedy takto:
    \sloppy
    \begin{lstlisting}
        public static BigDecimal cos(BigDecimal x) { 
            BigDecimal ans = new BigDecimal(0);
            // Suma
            for (BigDecimal n = BigDecimal.ZERO; n.compareTo(new BigDecimal(50)) <= 0; n = n.add(BigDecimal.ONE)) {
                // Citatel
                BigDecimal numerator = MINUSONE.pow(n.intValue());

                // Jmenovatel
                BigDecimal denominator = factorial(TWO.multiply(n));
                // Vzorec uvnitr sumy
                ans = ans.add((numerator.divide(denominator, 1000, 
                    RoundingMode.HALF_UP)).multiply(
                    x.pow((TWO.multiply(n)).intValue())));
            }
            // Vraceni vysledku
            return ans.setScale(50, RoundingMode.HALF_UP);  
        }
    \end{lstlisting}
    Tento algoritmus je nastaven na přesnost okolo $1\times{10^{-50}}$ s časem vypočítání okolo 10ms. Pro
    větší přesnost zvyšte \textit{n.compareTo(new BigDecimal(50))} hodnotu na vyšší.
    \subsubsection{Arctangens}
    Pro aproximaci arctangensu jsem opět použil taylorovy série, konkrétně jejich kombinaci, která bypadá takto:
    \begin{center}
        \begin{equation*}            
            arctan x = \begin{cases}
                \displaystyle\sum_{n=0}^{\infty}\frac{(-1)^n}{2n+1}x^{2n+1} & \text{$: |x|\leq 1$}\\
                \frac{\pi}{2}-\displaystyle\sum_{n=0}^{\infty}\frac{(-1)^n}{(2n+1)x^{2n+1}} & \text{$: x \geq 1$}\\
                -\frac{\pi}{2}-\displaystyle\sum_{n=0}^{\infty}\frac{(-1)^n}{(2n+1)x^{2n+1}} & \text{$: x \leq -1$}
            \end{cases}
        \end{equation*}
    \end{center}
    Výsledný algoritmus vypadá tedy takto:
    \sloppy
    \begin{lstlisting}
        public static BigDecimal arctan(BigDecimal x) {
            // Pokud je |x| <= 1
            if (x.abs().compareTo(BigDecimal.ONE) <= 0) {
                BigDecimal ans = new BigDecimal(0);
                // Suma
                for (BigDecimal n = BigDecimal.ZERO; n.compareTo(new BigDecimal(50)) <= 0; n = n.add(BigDecimal.ONE)) {
                    // Citatel
                    BigDecimal numerator = MINUSONE.pow(n.intValue());
                    // Jmenovatel
                    BigDecimal denominator = (TWO.multiply(n)).add(BigDecimal.ONE);
                    // Vzorec uvnitr sumy
                    ans = ans.add((numerator.divide(denominator, 1000, 
                        RoundingMode.HALF_UP)).multiply(
                        x.pow(((TWO.multiply(n)).add(
                        BigDecimal.ONE)).intValue())));

                    // Vraceni vysledku
                }
                return ans.setScale(50, RoundingMode.HALF_UP);
            // Jinak
            } else {
                BigDecimal ans = new BigDecimal(0);
                // Suma
                for (BigDecimal n = BigDecimal.ZERO; n.compareTo(new BigDecimal(50)) <= 0; n = n.add(BigDecimal.ONE)) {
                    // Citatel
                    BigDecimal numerator = MINUSONE.pow(n.intValue());
                    // Jmenovatel
                    BigDecimal denominator = 
                        ((TWO.multiply(n)).add(BigDecimal.ONE)).multiply(
                        x.pow(((TWO.multiply(n)).add(
                        BigDecimal.ONE)).intValue()));
                    // Vzorec uvnitr sumy
                    ans = ans.add((numerator.divide(denominator, 1000, RoundingMode.HALF_UP)));
                }
                // Pokud je x > 1
                if (x.compareTo(BigDecimal.ONE) > 0) {
                    return HALFPI.subtract(ans).setScale(50, RoundingMode.HALF_UP);
                // Pokud je x < 1
                } else if (x.compareTo(MINUSONE) < 0) {
                    return MINHALFPI.subtract(ans).setScale(50, RoundingMode.HALF_UP);
                // Jinak vrati chybny vysledek
                } else {
                    return ans.setScale(50, RoundingMode.HALF_UP);
                }
            }
        }
    \end{lstlisting}
    Tento algoritmus je nastaven na přesnost okolo $1\times{10^{-50}}$ s časem vypočítání okolo 10ms. Pro
    větší přesnost zvyšte \textit{n.compareTo(new BigDecimal(50))} hodnotu na vyšší.
    \subsubsection{Hyperbolický sinus}
    Pro aproximaci hyperbolického sinusu jsem použil opět taylorovu sérii.
    \begin{center}
        $sinh x = \displaystyle\sum_{n=0}^{\infty}\frac{x^{2n+1}}{(2n+1)!}$
    \end{center}
    Výsledný algoritmus vypadá tedy takto:
    \newline
    \begin{lstlisting}
        public static BigDecimal sinh(BigDecimal x) {
            BigDecimal ans = new BigDecimal(0);
            // Suma
            for (BigDecimal n = BigDecimal.ZERO; n.compareTo(new BigDecimal(50)) <= 0; n = n.add(BigDecimal.ONE)) {
                // Citatel
                BigDecimal numerator = x.pow((2*(n.intValue())+1));
                // Jmenovatel
                BigDecimal denominator = factorial((TWO.multiply(n)).add(BigDecimal.ONE));
                // Vzorec uvnitr sumy
                ans = ans.add(numerator.divide(denominator, 1000, RoundingMode.HALF_UP));
            }
            // Vraceni vysledku
            return ans.setScale(50, RoundingMode.HALF_UP);
        }
    \end{lstlisting}
    Tento algoritmus je nastaven na přesnost okolo $1\times{10^{-50}}$ s časem vypočítání okolo 10ms. Pro
    větší přesnost zvyšte \textit{n.compareTo(new BigDecimal(50))} hodnotu na vyšší.
    \subsubsection{Hyperbolický cosinus}
    Pro aproximaci hyperbolického cosinu je opět užita taylorova série.
    \begin{center}
        $cosh x = \displaystyle\sum_{n=0}^{\infty}\frac{x^{2n}}{(2n)!}$
    \end{center}
    Výsledný algoritmus vypadá tedy takto:
    \begin{lstlisting}
        public static BigDecimal cosh(BigDecimal x) {
            BigDecimal ans = new BigDecimal(0);
            // Suma 
            for (BigDecimal n = BigDecimal.ZERO; n.compareTo(new BigDecimal(50)) <= 0; n = n.add(BigDecimal.ONE)) {
                // Citatel
                BigDecimal numerator = x.pow(2*(n.intValue()));
                // Jmenovatel
                BigDecimal denominator = factorial(TWO.multiply(n));
                // Vzorec uvnitr sumy
                ans = ans.add(numerator.divide(denominator, 1000, RoundingMode.HALF_UP));
            }
            // Vraceni vysledku
            return ans.setScale(50, RoundingMode.HALF_UP);
        }
    \end{lstlisting}
    Tento algoritmus je nastaven na přesnost okolo $1\times{10^{-50}}$ s časem vypočítání okolo 10ms. Pro
    větší přesnost zvyšte \textit{n.compareTo(new BigDecimal(50))} hodnotu na vyšší.
    \subsection{Exponenciální funkce a přirozený logaritmus}
    \subsubsection{Exponenciální funkce}
    Exponenciální funkce ve tvaru $exp(x) = e^x$ je vypočítaná taylorovou sérií, která vypadá takto:
    \begin{center}
        $exp(x)=\displaystyle\sum_{n=0}^{\infty}\frac{x^n}{n!}$
    \end{center}
    Algoritmus pro vypočítaní vypadá tedy takto:
    \begin{lstlisting}
        public static BigDecimal exp(BigDecimal z) {
            BigDecimal ans = new BigDecimal(0);
            // Suma
            for (BigDecimal n = BigDecimal.ZERO; n.compareTo(new BigDecimal(150)) <= 0; n = n.add(BigDecimal.ONE)) {
                ans = ans.add((z.pow(n.intValue())).divide(factorial(n), 50, RoundingMode.HALF_UP));
            }
            // Navraceni vysledku
            return ans;
        }
    \end{lstlisting}
    Tento algoritmus je nastaven na přesnost okolo $1\times{10^{-50}}$ s časem vypočítání okolo 15ms. Pro
    větší přesnost zvyšte \textit{n.compareTo(new BigDecimal(50))} hodnotu na vyšší.
    \subsubsection{Přirozený logaritmus}
    Algorimizace přirozeného logarimu je složitá. Integrální vzorec je časově velice neefektivní, tudíž jsem
    se rozhodl použít lehce upravený Newtonův algoritmus. Vypadá takto:
    \begin{center}
        $log(x)=(a_{n-1}-\frac{e^{a_{n-1}}-n}{e^{a_{n-1}}})_{n=1}^{\infty}$; kde $a_0 = x$
    \end{center}
    Jeho výsledný algoritmus vypadá takto:
    \newline\newline
    \begin{lstlisting}
        public static BigDecimal log(BigDecimal x) {
            BigDecimal n = x; BigDecimal term;
            // Pokud x neni v definicnim oboru
            if (x.compareTo(BigDecimal.ZERO) <= 0) {
                return null;
            }
            // Vypocet posloupnosti
            for (int i = 1; i<=20; i++) {
                BigDecimal eToX = exp(x);
                term = eToX.subtract(n).divide(eToX, 50, RoundingMode.HALF_UP);
                x = x.subtract(term);
            }
            // Vraceni vysledku
            return x.setScale(50, RoundingMode.HALF_UP);
        }
    \end{lstlisting}
    Výsledek má přesnost okolo $1x10^(-50)$ a čas výpočtu je asi 500ms. Tento algoritmus není nejrychlejší,
    ale je velice přesný - pro větší přesnost musíte změnit \textit{$i \leq 20$} na větší číslo. Rychlost se
    tak sice zhorší, ale extrémně naroste přesnost. Momentální nastavení na \textit{20} je přepal, ale pro větší
    vstupy je potřeba větší přesnost.
    \subsection{Ostatní funkce}
    Pro některé algoritmy jsem potřeboval funkce, které nejsou tak důležité, aby měli svou vlastní sekci, nebo jich
    není mnoho.
    \subsubsection{Faktoriál}
    Výpočet tohoto faktoriálu je pouze pro přirozený čísla, není naprogramována gamma funkce, která by toto zgeneralizovala.
    Ovšem pro užití v taylorových seriích je potřeba pouze faktoriál pro přirozená čísla. Jeho předpis je tedy jednoduchý:
    \begin{center}
        $n! = x\times{(x-1)}\times{(x-2)}\times{...}\times{2}\times{1} = x\times{(n-1)!}$\newline
        $0! = 1$
    \end{center}
    Výsledný algortimus vypadá takto:
    \begin{lstlisting}
        public static BigDecimal factorial(BigDecimal n) {
            BigDecimal ans = n;
            // Pokud je vstup 0, vrati hodnotu 1
            if (n.compareTo(BigDecimal.ZERO) == 0) {
                return BigDecimal.ONE;
            }
            // Pokud je vstup zaporne cislo, vrati chybnou hodnotu
            if (n.compareTo(BigDecimal.ZERO) < 0) {
                return null;
            }
            // Jinak se provede cyklus pro vypocet factorialu
            for (BigDecimal k = new BigDecimal(1); k.compareTo(n) < 0; k = k.add(BigDecimal.ONE)) {
                ans = ans.multiply(n.subtract(k));
            }
            // Vrati hodnotu
            return ans;
        }
    \end{lstlisting}
    Tento algoritmus je velice rychlý - nemělo smysl měřit jeho rychlost. Navíc se jeho rychlost zpomaluje s větším číslem.
    \subsubsection{Sign funkce}
    Tato funkce je velice jednoduchá:
    \begin{center}
        \begin{equation*}
            sgn x = \begin{cases}
                -1 & \text{$: x < 0$}\\
                0 & \text{$: x = 0$}\\
                1 & \text{$: x > 0$}\\
            \end{cases}
        \end{equation*}
    \end{center}
    Její implementace je taktéž jednoduchá, ikdyž vypadá složitě.
    \begin{lstlisting}
        public static BigDecimal sign(BigDecimal x) {
            // Pokud je vstup mensi jak 0
            if (x.compareTo(BigDecimal.ZERO) < 0) {
                return new BigDecimal(-1);
            // Pokud je vstup roven 0
            } else if (x.compareTo(BigDecimal.ZERO) == 0) {
                return BigDecimal.ZERO;
            // Pokud je vstup vetsi jak 0
            } else {
                return BigDecimal.ONE;
            }
        }
    \end{lstlisting}
    \newpage
    %% Complex numbers section
    \section{ComplexNumber}
    V tomto souboru je vyřešena implementace kompexních čísel a jejich základních funkcí do Javy. Jako
    základní funkce se rozumí jejich operace, absolutní hodnota, sgn funkce a trigonometrické s hyperbolickými funkcemi. Dále
    tento soubor obsahuje i základní vypsání funkcí, statické hodnoty a jejich konstruktory.
    \subsection{Konstrukory, display funkce a převody.}
    V základu tu jsou definované 4 otevřené proměnné, které jsou používány ve funkcích. Převod mezi nimi je
    ukázán dále v tomto dokumentu.
    \begin{lstlisting}
        public BigDecimal REAL; // Realna hodnota komplexniho cisla
        public BigDecimal IMG; // Imaginarni hodnota komplexniho cisla
        public BigDecimal R; // Polomer komplexniho cisla v goniometrickem tvaru
        public BigDecimal PHI; // Uhel ke komplexnimu cislu v goniometrickem tvaru
    \end{lstlisting}
    \subsubsection{Konstruktory}
    Je zde definovaných několik konstruktorů, které vytvoří toto komplexní číslo. Prvním z nich je tzv.
    prázdný konstruktor, který nastaví hodnotu $0+0i$.
    \begin{lstlisting}
        public ComplexNumber() {
            REAL = BigDecimal.ZERO; IMG = BigDecimal.ZERO;
        }
    \end{lstlisting}
    Dále jsou konstruktory, kteří přijímají jakékoliv číselné hodnoty po dvojicích. Pordporované hodnoty jsou
    \textit{int, double, ,float, BigInteger a BigDecimal}. Příkladný konstruktor tedy vypadá takto:
    \begin{lstlisting}
        public ComplexNumber(double a, double b) {
            REAL = new BigDecimal(a); IMG = new BigDecimal(b);
        }
    \end{lstlisting}
    Vrátí tedy hodnotu $a+bi$, kde $a$ i $b$ jsou $double$ hodnoty. Tyto konstruktory jsou nastaveny na dvojice -
    to znamená že musíte do konstruktoru vložit dvě hodnoty stejného typu. \newline
    Jejich zavolání vypadá následovně:
    \begin{lstlisting}
        ComplexNumber a = new ComplexNumber(); // Prazdny konstruktor
        ComplexNumber b = new ComplexNumber(1, 2); // Konstruktor o hodnote "1+2i"
    \end{lstlisting}
    \subsubsection{Display funkce}
    Display funkce vypíše hezky formátovaně komplexní tvar čísel. Jsou nastaveny na podmínky, aby dokázali měnit
    znaménko mezi reálnou a komplexní hodnotou, jinak by se čísla buď na sebe nalepila či by se vypsaly dvě znaménka
    vedle sebe. Případně vypíší reálnou či imaginární část, pokud jsou rovny nule.
    \begin{lstlisting}
        public void display() {
            // Pokud je realna cast rovna 0, vypise pouze imaginarni cast
            if (REAL.compareTo(BigDecimal.ZERO) == 0) {
                System.out.println(
                    IMG.toString()+"i"
                );
                return;
            // Pokud je imaginarni cas rovna 0, vypise pouze realnou cast
            } else if (IMG.compareTo(BigDecimal.ZERO) == 0) {
                System.out.println(
                    REAL.toString()
                );
                return;
            }
            // Zmeni znamenko podle velikosti imaginarni casti
            if (IMG.compareTo(BigDecimal.ZERO) < 0) {
                System.out.println(
                    REAL.toString()+"-"+IMG.abs().toString()+"i"
                );
                return;
            } else {
                System.out.println(
                    REAL.toString()+"+"+IMG.toString()+"i"
                );
                return;
            }
        }
    \end{lstlisting}
    Zavolání těchto funkcí je tedy následovné:
    \begin{lstlisting}
        ComplexNumber a = new ComplexNumber(5, 4);
        a.display();
        // 5+4i
    \end{lstlisting}
    \subsubsection{Převod do goniometrického tvaru}
    Goniomtrický tvar se skládá z poloměru $r$ a úhlu $\phi$ ke komplexnímu číslu $z = x+yi$. Výpočet poloměru je následovný:
    \newline
    \begin{center}
        $r = |z| = \sqrt{x^2+y^2}$
    \end{center}
    A výpočet úhlu $\phi$ je:
    \begin{center}
        \begin{equation*}
            \phi = arg(z) = \begin{cases}
                2arctan(\frac{y}{\sqrt{x^2+y^2}+x}) & \text{if $y \neq 0$}\\
                \pi & \text{if $x < 0$ and $y = 0$}\\
                0 & \text{if $y = 0$ and $x > 0$}\\
                undefined & \text{if $x = 0$ and $y = 0$}
            \end{cases}
        \end{equation*}
    \end{center}
    Tento tvar je použit při výpočtu určitých operacích, napřiklad při exponenciální funkce.
    \newline
    Algoritmus na vytvoření těchto proměnných je následovný:
    \begin{lstlisting}
        public void polar_conversion() {
            // Polomer
            R = (REAL.pow(2).add(IMG.pow(2))).sqrt(new MathContext(50));
            // Fi hodnota
            // Treti podminka z predesleho vzorce
            if (REAL.compareTo(BigDecimal.ZERO) != 0 && IMG.compareTo(BigDecimal.ZERO) == 0) {
                PHI =  BigDecimal.ZERO;
            // Prvni podminka
            } else if (REAL.compareTo(BigDecimal.ZERO) > 0 || IMG.compareTo(BigDecimal.ZERO) != 0) {
                BigDecimal TWO = new BigDecimal(2);
                PHI = (
                    TWO.multiply(BigDecimalMath.arctan(
                    IMG.divide(R.add(REAL), 50, RoundingMode.HALF_UP)))
                );
                PHI = PHI.setScale(50, RoundingMode.HALF_UP);
            // Druha podminka
            } else if (REAL.compareTo(BigDecimal.ZERO) < 0 && IMG.compareTo(BigDecimal.ZERO) == 0) {
                PHI = new BigDecimal(Math.PI);
                PHI = PHI.setScale(50, RoundingMode.HALF_UP);
            // Posledni podminka
            } else {
                PHI = null;
            }
        }
    \end{lstlisting}
    Tyto hodnoty nejsou vytvořeny vždy, ale ve funkcích, kde jsou potřeba se tato funkce sama zavolá a
    hodnoty se vypořítají.
    \subsection{Konstanty a čísla}
    Opět, jako u \textit{BigDecimalMath} jsou zde vytvořeny nějaké základní konstanty a čísla, pro jednodušší
    použití.
    \begin{lstlisting}
        // Nulova hodnota
        public static final ComplexNumber ZERO = new ComplexNumber(); 
        // Hodnota pro cislo jedna
        public static final ComplexNumber ONE = new ComplexNumber(1, 0);
        // Hodnota pro cislo minus jedna
        public static final ComplexNumber MINUSONE = new ComplexNumber(-1, 0);
        // Hodnota pro cislo I
        public static final ComplexNumber I = new ComplexNumber(0, 1);
        // Hodnota pro cislo 0.5I
        public static final ComplexNumber HALFI = new ComplexNumber(0, 0.5);
        // Hodnota pro cislo Pi/2
        public static final ComplexNumber HALFPI = new ComplexNumber(BigDecimalMath.PI.divide(BigDecimalMath.TWO, 1000, RoundingMode.HALF_UP));
    \end{lstlisting}
    Opět některé tyto hodnoty mohou vypadat nesmyslně, proč jsou definovány, ale jsou potřebovat v některých algoritmech.
    \subsection{Základní operace}
    Definujme si 2 komplexní čísla. Číslo $c = a+bi$ a $z = x+yi$. Výsledek budu označovat za číslo $\alpha$
    \subsubsection{Sčítání a odčítání}
    Tyto operace jsou velice jednoduché a intuitivní.
    Operace sčítání:
    \begin{center}
        $\alpha = c+z = (a+bi)+(x+yi) = (a+x)+(b+y)i$
    \end{center}
    \begin{lstlisting}
        public ComplexNumber add(ComplexNumber b) {
            ComplexNumber ans = new ComplexNumber();
            ans.REAL = REAL.add(b.REAL);
            ans.IMG = IMG.add(b.IMG);
            return ans;
        }
    \end{lstlisting}
    Operace odčítání:
    \begin{center}
        $\alpha = c-z = (a+bi)-(x+yi) = (a-x)+(b-y)i$
    \end{center}
    \begin{lstlisting}
        public ComplexNumber subtract(ComplexNumber b) {
            ComplexNumber ans = new ComplexNumber();
            ans.REAL = REAL.subtract(b.REAL);
            ans.IMG = IMG.subtract(b.IMG);
            return ans;
        }
    \end{lstlisting}
    \subsubsection{Násobení}
    Tato operace je opět velice jednoduchá:
    \begin{center}
        $\alpha = c\times{z} = (a+bi)\times{(x+yi)} = (ax-by)+(ay+bx)i$
    \end{center}
    \begin{lstlisting}
        public ComplexNumber multiply(ComplexNumber b) {
            ComplexNumber ans = new ComplexNumber();
            ans.REAL = (REAL.multiply(b.REAL)).subtract(IMG.multiply(b.IMG));
            ans.IMG = (REAL.multiply(b.IMG)).add(IMG.multiply(b.REAL));
            return ans;
        }
    \end{lstlisting}
    \subsubsection{Dělení}
    Tato operace je trochu složitější:
    \begin{center}
        $\frac{c}{z} = c\times{\frac{1}{z}} = \frac{(ax+by)+(bx-ay)i}{x^2+y^2}$
    \end{center}
    Její algoritmus:
    \begin{lstlisting}
        public ComplexNumber divide(ComplexNumber b) {
            ComplexNumber ans = new ComplexNumber();
            // Pokud je jmenovatel roven 0, vrati chybnou hodnotu
            if (b.REAL.compareTo(BigDecimal.ZERO) == 0 && b.IMG.compareTo(BigDecimal.ZERO) == 0) {
                return null;
            }
            // Vlastni algoritmus
            ans.REAL = ((REAL.multiply(b.REAL)).add(IMG.multiply(b.IMG))).divide(
                (b.REAL.pow(2)).add(b.IMG.pow(2)), 50, 
                RoundingMode.HALF_UP);
            ans.IMG = ((IMG.multiply(b.REAL)).subtract(REAL.multiply(b.IMG))
                ).divide((b.REAL.pow(2)).add(b.IMG.pow(2)), 50, 
                RoundingMode.HALF_UP);
            // Vraceni vysledku
            return ans;
        }
    \end{lstlisting}
    % \newpage
    % %%% Resources
    % \subsection{Zdroje}
    % \underline{Zdroj pro aproximace trigonometrických funkcí:} \newline
    % \indent \textbf{Wikipedia:} Taylor series \newline
    % \indent \textbf{Wikiproof:} Power series expansion for real arctangnet function \newline
    % \underline{Zdroj pro aproximace hyperbolických funkcí:} \newline
    % \indent \textbf{Wikipedia:} Taylor series \newline
    % \underline{Zdroj pro exponenciální funkci:} \newline
    % \indent \textbf{Wikipedia:} Taylor series \newline
    % \underline{Zdroj pro přirozený logaritmus:} \newline
    % \indent \textbf{Wikipedia:} Natural logarithm high precision \newline
    % \underline{Zdroj pro faktoriál:} \newline
    % \indent \textbf{Wikipedia:} Factorial
    
\end{document}